\input{beamer_setup.tex}
\title[Simbe]{Introducing Simbe for Technical Slides}
\subtitle{Simbe}
\author[JP Onnela]{JP Onnela}
\date[July 10, 2021]{July 10, 2021}
\institute[]{Department of Biostatistics \\ Harvard University}
\frame{\titlepage}

**Introducing Simbe
-Simbe is an ultralight markup language for math / code heavy slides
-I wrote it in 2013 when I had to prepare 600+ slides for teaching a new course
-PowerPoint and Keynote were not feasible options for technical slides
-LaTeX Beamer has too much markup overhead for simple functionalities
-Simbe is a very simple LaTeX preprocessor written in Python 3
-It converts a Simbe file to a standard latex file which is compiled to PDF slides
-Simbe is short for Simple LaTeX Beamber

**Functionality of Simbe
-Simbe makes the following LaTeX / Beamer operations easy
    -Bullets
    -Equations
    -Figures
    -Code with syntax highlighting
-These cover 99\% of my needs, but it's really just LaTeX, so you can do anything
-This is a famous equation:
--
E=mc^2
--

**Figures
-Computers are now used everywhere in science
---
my_figure.pdf, 0.7
-This is a serious computer.
---

**Code with Syntax Highlighting
-Python is increasingly used in research settings
-Check out my HarvardX course ``Using Python for Research''
-Here's a simple Python program with syntax highlighting:
----
from math import pi
print(pi)
----

**Code with Syntax Highlighting
-Some programs are more complicated
-In some cases it's better to place a program in its own file
-It's especially heplful if you want to execute code on your slides
-Here's Python code for generating the Fibonacci sequence
-----my_code.py-----

